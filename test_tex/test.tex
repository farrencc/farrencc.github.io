\documentclass{article}
\usepackage{amsmath,amsfonts,amssymb,amsthm}
\usepackage{braket}
\usepackage{bbold}
\usepackage{physics}
\usepackage{graphicx}
\usepackage[utf8]{inputenc}
\usepackage[T1]{fontenc}
\usepackage{mathtools}
\usepackage[thinc]{esdiff}
\usepackage{bigints}
\usepackage{enumerate}
\usepackage{wasysym}
\usepackage{pythonhighlight}
\usepackage{caption}
\usepackage{subcaption}
\usepackage{esint}
% \usepackage[top=1in,bottom=1in,left=1in,right=in]{geometry}
\usepackage{mathtools}
\usepackage{cancel}
\usepackage{hyperref}
\usepackage{mathrsfs}
\usepackage{algorithm}
\usepackage{algpseudocode}
\DeclareMathOperator{\arccosh}{arcCosh}
\DeclareMathOperator{\arcsinh}{arcsinh}
\DeclareMathOperator{\arctanh}{arctanh}
\DeclareMathOperator{\arcsech}{arcsech}
\DeclareMathOperator{\arccsch}{arcCsch}
\DeclareMathOperator{\arccoth}{arcCoth}
\renewcommand{\vec}{\mathbf}
\newcommand{\gvec}{\boldsymbol}
\newenvironment{claim}[1]{\par\noindent\underline{Claim:}\space#1}{}
\newenvironment{claimproof}[1]{\par\noindent\underline{Proof:}\space#1}{\hfill $\blacksquare$}
\makeatletter
\def\@seccntformat#1{%
  \expandafter\ifx\csname c@#1\endcsname\c@section\else
  \csname the#1\endcsname\quad
  \fi}
\def\@seccntformat#1{%
  \expandafter\ifx\csname c@#1\endcsname\c@subsection\else
  \csname the#1\endcsname\quad
  \fi}
\let\latexl@section\l@section
\def\l@section#1#2{\begingroup\let\numberline\@gobble\latexl@section{#1}{#2}\endgroup}
\let\latexl@subsection\l@subsection
\def\l@subsection#1#2{\begingroup\let\numberline\@gobble\latexl@subsection{#1}{#2}\endgroup}
\makeatother
\newcommand{\dbar}{d\hspace*{-0.08em}\bar{}\hspace*{0.1em}}
\title{}
\author{Casey Farren-Colloty (21365022)}
\date{}
\begin{document}
\maketitle

\begin{itemize}
% \tightlist
\item
  \hyperref[about]{About}
\item
  \hyperref[research]{Research}
\item
  \hyperref[teaching]{Teaching}
\item
  \hyperref[links]{Links}
\end{itemize}

\section{Casey Farren-Colloty}\label{casey-farren-colloty}

Hi, my name is Casey although most people call me {Cas}

\href{https://github.com/yourusername}{{GitHub}}
\href{https://linkedin.com/in/yourusername}{{LinkedIn}}

\phantomsection\label{featured}
\subsection{Favourite Equation / Quote at the
Moment}\label{favourite-equation-quote-at-the-moment}

\phantomsection\label{equation}
\textbf{"...these lectures are about as theoretical as they come.
We\textquotesingle re not actually going to measure anything. Just
pretend."} -
\emph{\href{http://www.damtp.cam.ac.uk/user/tong/qhe/qhe.pdf}{David
Tong}}

There is an argument to be made that physics is at risk of becoming far
too overly specialised. Indeed, this has been becoming a problem since
the days of Fermi - as the last physicist to be considered both an
experimentalist and a theorist. Though perhaps the advent of
computational physics, allowing us to test theory in a way that was
previously impossible, will allow for a new generation of physicists to
be both theorists and pseudo-experimentalists. But this quote is also
just funny.

\phantomsection\label{about}
\section{About Me}\label{about-me}

"I\textquotesingle m a third-year theoretical physics student at Trinity
College Dublin, driven by a deep curiosity to understand the fundamental
principles that govern our universe as well as the application of those
principles to the broader human society. I find areas that offer both
profound insights into nature\textquotesingle s workings and practical
applications for a sustainable future particularly engaging. Beyond
research, I\textquotesingle m passionate about making science accessible
to everyone. As Seminar Director of the
\href{https://www.tpsa.ie/homepage}{Theoretical Physics Student
Association of Ireland} and Director of
\href{https://tpsa.ie/problemsolving}{The Problem Solving Association
CLG}, I\textquotesingle m involved in initiatives in science
communication and education. From creating physics lecture series and
teaching at the \href{https://www.dcu.ie/ctyi}{Centre for Talented Youth
(CTYI)} to working on projects furthering democratic society through
utilising physics-based techniques, I believe in sharing the
problem-solving mindset of theoretical physics to address challenges
across disciplines.

\subsection{Research}\label{research}

\subsubsection{2024 - Oxford University
Internship}\label{oxford-university-internship}

Selected for the prestigious
\href{https://www.physics.ox.ac.uk/research/subdepartment/rudolf-peierls-centre-theoretical-physics/undergraduate-research}{Rudolf
Peierls Centre for Theoretical Physics UROP}, I investigated Renormalon
effects in particle physics. Working under Prof. Gavin P. Salam FRS,
Prof. Fabrizio Caola, Jack Oliver Helliwell, and Silvia Zanoli, I
studied \textbackslash( e\^{}- e\^{}+ \textbackslash) annihilation
events. My research focused on
\href{https://arxiv.org/abs/hep-ph/0312283}{event shape observables},
using the \href{https://gitlab.com/panscales/panscales-0.X}{PanScales}
software to simulate electron-positron annihilation and analyze
geometric properties of quark anti-quark pair events, including
corrections for soft gluons and gluers.

{Particle Physics} {Quantum Field Theory} {Numerical Methods}

\subsubsection{2023 - Electoral Redistricting
Project}\label{electoral-redistricting-project}

Democratic systems face a growing crisis of trust. This project tackles
the challenge of fair electoral redistricting using computational
methods from physics. Following a successful
\href{https://www.maths.tcd.ie/~tristan/SSG/index.html}{hackathon}
organized by TPSA and TCD\textquotesingle s School of Mathematics, I
developed a genetic algorithm approach to redistribute Irish electoral
constituencies. This complemented a physics-based algorithm developed by
\href{https://www.maths.tcd.ie/~rcampion/}{Ruaidhrí Campion}, forming a
comprehensive dual approach to electoral fairness. The work continues
through The Problem Solving Association CLG, aiming to restore
institutional trust through transparent, algorithmic solutions.

{Computational Physics} {Social Systems} {Algorithmic Design}

\phantomsection\label{teaching}
\subsection{Teaching}\label{teaching}

I\textquotesingle m a firm believer of the importance of putting effort
in to show the younger generation how interesting the world of science
can be. As well as to show them that participating in this world is an
achievable goal. The opportunity arose in the academic year 2023/24 to
do just that and teach at the Centre for Talented Youth Ireland.
Specifically, the 8-12 year old course. Here are some of the materials
for these courses:

\subsubsection{Superhero Science}\label{superhero-science}

Introduction to a variety of topics in science and engineering such as
magnetism, genetics, and material science.

\href{https://github.com/farrencc/ctyi_superhero_science}{View Course
Materials →}

\subsubsection{Astronomy}\label{astronomy}

Fundamentals of astronomy and related areas of physics. Topics included:
Special / General Relativity, Stars, Galaxies, Stellar Evolution, and
Rockets.

\href{https://github.com/farrencc/ctyi_astronomy}{View Course Materials
→}

\phantomsection\label{links}
\subsection{Relevant Links}\label{relevant-links}

\subsubsection{GitHub}\label{github}

Access my research code, teaching materials, and personal projects.

\href{https://github.com/farrencc}{Visit Profile →}

\subsubsection{The TPSA}\label{the-tpsa}

Learn more about our current projects and more. Both from the Student
Association and Non-Profit.

\href{https://www.tpsa.ie}{Visit Page →}

© 2024 Casey Farren-Colloty \textbar{} Theoretical Physics Student and
Non-Profit Director

\href{mailto:casey@tpsa.ie}{Email}
\href{https://github.com/farrencc}{GitHub}

\end{document}